\documentclass{beamer}

 \pdfmapfile{+sansmathaccent.map}


\mode<presentation>
{
  \usetheme{Madrid}      % or try Darmstadt, Madrid, Warsaw, ...
%   \usecolortheme{beaver} % or try albatross, beaver, crane, ...
  \usefonttheme{serif}  % or try serif, structurebold, ...
  \setbeamertemplate{navigation symbols}{}
  \setbeamertemplate{caption}[numbered]
} 


\renewcommand{\familydefault}{\rmdefault}


\usepackage{amsmath}
\usepackage{mathtools}

\DeclareMathOperator*{\argmin}{arg\,min}


\usepackage{subcaption}


\title{ Linear Algebra for LTI systems}
\author{by Sergei Savin}
\centering
\date{Spring 2020}



\begin{document}
\maketitle


\begin{frame}{Content}

\begin{itemize}
\item Which states are possible?
\item How to check if the state is possible?
\item Check if affine system is stabilizable
\item Criteria without orthonormal basis
\item Checking if an arbitrary point can be stabilized
\item All stabilizable points
\end{itemize}

\end{frame}




\begin{frame}{Which states are possible?}
% \framesubtitle{Parameter estimation}
\begin{flushleft}

Consider discrete autonomous LTI system:
%
\begin{equation}
\label{DLTI}
    \mathbf x_{i+1} = \mathbf A \mathbf x_i
\end{equation}
%
where $\mathbf x \in \mathbb{R}^n$ is the state of the system, expressed in the basis $\mathbf O$. What possible values can $\mathbf x_{i+1}$ attain?

\bigskip

From the \eqref{DLTI} follows that all possible $\mathbf x_{i+1}$ are in the \emph{column space} of $\mathbf A$.

\[
\mathbf x_{i+1} \in \mathcal{X} = \text{col} (\mathbf A) \subseteq \mathbb{R}^n
\]

% \bigskip

% If $\mathcal{X} \neq \mathbb{R}^n$, then $\exists \mathcal{N}$, s.t. $\mathcal{X} \bigoplus \mathcal{N} = \mathbb{R}^n$, where $\mathcal{N}$ is called null space of $\mathbf A$: 
% \[
% \mathcal{N} = \text{null} (\mathbf A)
% \]

\end{flushleft}
\end{frame}



\begin{frame}{How to check if the state is possible?}
\framesubtitle{Part 1}
\begin{flushleft}

How can we tell if a particular value of $\mathbf x_{i+1}$ is possible? More general, having an the equation $\mathbf h = \mathbf A \mathbf x$, how do check that for a particular $\mathbf h$, $\exists \mathbf x$, s.t. the equality is satisfied?

\bigskip

Let $\mathbf P$ be an orthonormal basis in the column space of $\mathbf A$:
\[
\mathbf P = \text{orth} (\mathbf A)
\]
Columns of $\mathbf P$ span the column space of $\mathbf A$, or in other words, the column spaces of $\mathbf A$ and $\mathbf P$ and the same: $\mathcal{X} = \text{col} (\mathbf A) = \text{col} (\mathbf P)$.

\end{flushleft}
\end{frame}



\begin{frame}{How to check if the state is possible?}
\framesubtitle{Part 2}
\begin{flushleft}

Vector $\mathbf h$ is expressed in the basis $\mathbf O$, which we can denote as $\mathbf h^{\mathbf O}$. Then, we can find coordinates of the projection of $\mathbf h$ onto $\mathcal{X}$ in the basis $\mathbf P$:

\[
\mathbf h_p^{\mathbf P} = \mathbf P^\top \mathbf h^{\mathbf O}
\]

In the basis $\mathbf O$, vector $\mathbf h_P$ is given by the equation $\mathbf h_p^{\mathbf O} = \mathbf P \mathbf h_p^{\mathbf P}$:

\[
\mathbf h_p^{\mathbf O} =\mathbf P  \mathbf P^\top \mathbf h^{\mathbf O}
\]

\bigskip

Notice, that if vector $\mathbf h$ lies in the column space $\mathcal{X}$, its projection onto $\mathcal{X}$, namely $\mathbf h_p^{\mathbf O}$, should be equal to $\mathbf h$. Let us define projection residual $\mathbf e$:
\[
\mathbf e^{\mathbf O} = \mathbf h_p^{\mathbf O} - \mathbf h^{\mathbf O}
\]

Therefore we con formulate that $\mathbf h \in \mathcal{X}$ iff $\mathbf e = 0$.


\end{flushleft}
\end{frame}




\begin{frame}{How to check if the state is possible?}
\framesubtitle{Part 3}
\begin{flushleft}

Now we can formulate the condition for $\mathbf h \in \mathcal{X} = \text{col} (\mathbf A)$:
\[
\mathbf P \mathbf P^\top \mathbf h - \mathbf h = 0
\]
or:
\[
(\mathbf P \mathbf P^\top - \mathbf I) \mathbf h = 0
\]
where $\mathbf P = \text{orth} (\mathbf A)$

\end{flushleft}
\end{frame}




\begin{frame}{Check if affine system is stabilizable}
\framesubtitle{Part 1}
\begin{flushleft}

Consider affine linear system:
\begin{equation}
\label{eq:Affine}
    \dot{\mathbf x} = \mathbf A \mathbf x + \mathbf B \mathbf u + \mathbf c
\end{equation}

Can it be stabilized? Let the control law be give affine:
\[
\mathbf u = -\mathbf K \mathbf x - \mathbf u_0
\]
We want to find such $\mathbf K$ and $\mathbf u_0$ that:

\begin{itemize}
    \item dynamics of the system becomes $\dot{\mathbf x} =  (\mathbf A - \mathbf B \mathbf K) \mathbf x$.
    \item $\mathbf A - \mathbf B \mathbf K < 0$.
\end{itemize}

\end{flushleft}
\end{frame}



\begin{frame}{Check if affine system is stabilizable}
\framesubtitle{Part 2}
\begin{flushleft}

In order for the 
$\begin{cases}
    \dot{\mathbf x} = \mathbf A \mathbf x + \mathbf B \mathbf u + \mathbf c \\
    \mathbf u = -\mathbf K \mathbf x - \mathbf u_0
\end{cases}$
to become
$\dot{\mathbf x} =  (\mathbf A - \mathbf B \mathbf K) \mathbf x$,
the following equality needs to hold:
\[
\mathbf B \mathbf u_0 = \mathbf c
\]

In other words, $\mathbf c$ needs to be in the column space of $\mathbf B$. And we know the conditions we need to check:

\[
(\mathbf P \mathbf P^\top - \mathbf I) \mathbf c = 0,
\]
where $\mathbf P = \text{orth} (\mathbf B)$

\end{flushleft}
\end{frame}


\begin{frame}{Criteria without orthonormal basis}
% \framesubtitle{Part 2}
\begin{flushleft}

Finding an orthonormal basis might be excessive for this task, and there is a more direct way of checking if for a given $\mathbf h$, $\exists \mathbf x$, s.t.  $\mathbf h = \mathbf A \mathbf x$.

If there exists $\mathbf x^*$ such that $\mathbf h = \mathbf A \mathbf x^*$, then it can be found as:

\[
\mathbf x^* = \text{argmin} \ || \mathbf h - \mathbf A \mathbf x ||
\]

Solution to this \emph{linear least squares problem} is a psuedo inverse:
\[
\mathbf x^* = \mathbf A^+ \mathbf h
\]

Therefore, projection residual equation can be written as:
\[
\mathbf e = \mathbf A \mathbf x^* - \mathbf h
\]

Same as before, original question then comes down to proving that $\mathbf e = 0$:
\[
\mathbf A \mathbf A^+ \mathbf h - \mathbf h = 0
\]
or
\[
(\mathbf A \mathbf A^+ - \mathbf I) \mathbf h  = 0
\]


\end{flushleft}
\end{frame}


\begin{frame}{Checking if an arbitrary point can be stabilized}
% \framesubtitle{Part 2}
\begin{flushleft}

Consider a linear system:
\begin{equation}
\begin{cases}
    \dot{\mathbf x} = \mathbf A \mathbf x + \mathbf B \mathbf u \\
    \mathbf u = -\mathbf K (\mathbf x - \mathbf x_0) - \mathbf u_0
\end{cases}
\end{equation}

Is point $\mathbf x_0$ stabilizable? What this question means, is can we find such $\mathbf K$ and $\mathbf u_0$ that the system converges to $\mathbf x_0$?

\bigskip

Assume we are at the point $\mathbf x_0$. Then $\mathbf K (\mathbf x - \mathbf x_0) = 0$. Can we make sure we stay at this point? If not, it is not stabilized.

This comes down to solving equation:

\[
\mathbf A \mathbf x_0 - \mathbf B \mathbf u_0 = 0
\]

In other words, we need to find if $\mathbf A \mathbf x_0$ is in the column space of $\mathbf B$. Here is the criterion:
\[
(\mathbf B \mathbf B^+ - \mathbf I) (\mathbf A \mathbf x_0)  = 0
\]

\end{flushleft}
\end{frame}



\begin{frame}{All stabilizable points}
% \framesubtitle{Part 2}
\begin{flushleft}

For the system from the previous example we can easily find all points that are stabilizable. For that we consider equation:
\[
\mathbf A \mathbf x_0 - \mathbf B \mathbf u_0 = 0
\]
but this time we make both $\mathbf x_0$ and $\mathbf u_0$ our variables. This system has a nontrivial solution if matrix $[\mathbf A, \ -\mathbf B]$ has a nontrivial \emph{null space}.

\bigskip

Let $\mathbf z = \begin{bmatrix} \mathbf x_0 \\ \mathbf u_0 \end{bmatrix}$. And let $\mathbf N$ be a basis in the null space of $[\mathbf A, \ -\mathbf B]$:

\[
\mathbf N = \text{null}([\mathbf A, \ -\mathbf B])
\]

Then any combination of its columns produces a vector $\mathbf z$, which is stabilizable:
\[
\begin{bmatrix} \mathbf x_0 \\ \mathbf u_0 \end{bmatrix} = \mathbf z = 
\mathbf N \mathbf r
\]
where $\mathbf r$ is a random vector.

\end{flushleft}
\end{frame}



\begin{frame}
\centerline{Lecture slides are available via Moodle.}
\bigskip
\centerline{You can help improve these slides at:}
\centerline{\url{https://github.com/SergeiSa/Linear-Control-Slides-Spring-2020}}
\bigskip
\centerline{Check Moodle for additional links, videos, textbook suggestions.}
\end{frame}

\end{document}
