\documentclass{beamer}
\usetheme{Madrid}


 \pdfmapfile{+sansmathaccent.map}
 
\renewcommand{\familydefault}{\rmdefault}


\usepackage{amsmath}
\usepackage{mathtools}

\DeclareMathOperator*{\argmin}{arg\,min}

\title{Parameter estimation}
\author{by Sergei Savin}
\centering
\date{Spring 2020}


\begin{document}
\maketitle


\begin{frame}{Content}
\begin{itemize}
\item Parameter estimation
\begin{itemize}
\item Problem statement
\item Parameter uncertainty sources
\item Problem statement - more precise
\item System Linear in Parameters
\item Example
\end{itemize}
\item Least Squares in parameter estimation
\begin{itemize}
\item Problem statement
\item Solution
\item Multiple measurements
\end{itemize}
\end{itemize}
\end{frame}

\begin{frame}{Parameter estimation}
\framesubtitle{Problem statement}
\begin{flushleft}

Assume we have a system, whose model is defined in terms of parameters $\theta$:

\[
\dot {\mathbf x} = \mathbf f(\mathbf x, \mathbf u, \theta)
\]

Task: knowing relations $\mathbf f(\mathbf x, \mathbf u) = \mathbf f(\mathbf x, \mathbf u, \theta)$, find a stabilizing control law.

\bigskip

In case of a linear system, the dynamics is given in the form:
\[
\dot {\mathbf x} = \mathbf A(\theta) \mathbf x + \mathbf B(\theta) \mathbf u
\]

and the task is: knowing relations $\mathbf A = \mathbf A(\theta)$ and $\mathbf B = \mathbf B(\theta)$, find a stabilizing control law.

\end{flushleft}
\end{frame}

\begin{frame}{Parameter estimation}
\framesubtitle{Parameter uncertainty sources}
\begin{flushleft}

There are a number of reasons parameters of the model can be unknown:

\begin{itemize}
\item  Parameters that can’t be measured.
\item  Parameters change with time.
\item  Unmodeled dynamics.
\item  Unknown forces.
\item  Dynamic changes in the system (picking up a load, etc).
\end{itemize}

\end{flushleft}
\end{frame}


\begin{frame}{Parameter estimation}
\framesubtitle{Problem statement - Example 1}
\begin{flushleft}

Unfortunately, in order to treat the problem we need to assume that we can, directly or indirectly, measure higher order derivatives.

\bigskip

\emph{Example 1}

Let us consider spring-damper system: $m \ddot q + \mu \dot q + k q = u$. Introducing notation $y = u$ we arrive at:
\[
y = m \ddot q + \mu \dot q + k q
\]

\end{flushleft}
\end{frame}


\begin{frame}{Parameter estimation}
\framesubtitle{Problem statement - Example 2}
\begin{flushleft}

\emph{Example 2}

Consider a general mechanical system:
\[
\mathbf H \ddot {\mathbf q} + \mathbf c(\dot {\mathbf q}, \mathbf q) = \mathbf u
\]

This system is \emph{linear} with respect to parameters such as $m_i$, $l_i l_j$, $l_i^2$, $m_i l_i^2$, $m_i g$, etc., where $l_i$ are lengths, $m_i$ are masses, $g$ is the gravitational acceleration. Same as before, we make $y = u$. Therefore we can rewrite it as follows:
\[
\mathbf y = \mathbf Y (\ddot {\mathbf q}, \dot {\mathbf q}, \mathbf q) \theta
\]
where matrix $\mathbf Y$ can be a nonlinear function of the coordinates $\mathbf q$ and their derivatives. It is called a \emph{regressor}.


\end{flushleft}
\end{frame}




\begin{frame}{Parameter estimation}
\framesubtitle{System Linear in Parameters}
\begin{flushleft}

From here on we assume that we measure $\mathbf y$ and we know that $\mathbf y$ is a function of state and it's derivatives, and parameters $\theta$. 

Moreover, we assume that $\mathbf y$ is \emph{linear with respect to parameters} $\theta$:

\[
\mathbf y = \mathbf M (\mathbf x, \dot {\mathbf x}) \theta
\]

where $\mathbf M$ will be referred to as regressor matrix.

\bigskip

We introduce definitions:

\begin{itemize}
    \item $\Tilde{\theta}$ - estimated value of parameters $\theta$;
    \item $\varepsilon_{\theta} = \theta - \Tilde{\theta}$ - parameter estimation error;
    \item $\mathbf e_{\theta} = \mathbf y - \mathbf M \Tilde{\theta}$ - parameter estimation output error.
\end{itemize}

\end{flushleft}
\end{frame}

\begin{frame}{Parameter estimation}
\framesubtitle{Example}
\begin{flushleft}

Going back to the spring-damper system $y = m \ddot q + \mu \dot q + k q$, we can denote:

\begin{itemize}
    \item $\mathbf M = [\ddot q \; \dot q \; q]$ - regressor matrix;
    \item $\theta = [m \; \mu \; k]^\top$ - parameters;
\end{itemize}

Thus we get:

\[
y = [\ddot q \; \dot q \; q] [m \; \mu \; k]^\top = \mathbf M \theta
\]

Another example is a pendulum: $y = m l^2 \ddot \varphi + m g l sin(\varphi)$. We can denote:
\begin{itemize}
    \item $\mathbf M = [\ddot \varphi \; sin(\varphi)]$ - regressor matrix;
    \item $\theta = [m l^2 \; m g l]^\top$ - parameters;
\end{itemize}
Thus we get:

\[
y = [\ddot \varphi \; sin(\varphi)] [m l^2 \; m g l]^\top = \mathbf M \theta
\]


\end{flushleft}
\end{frame}



\begin{frame}{Least Squares in parameter estimation}
\framesubtitle{Problem statement}
\begin{flushleft}

We want to minimize parameter estimation error $\varepsilon_{\theta}$. However, we do not measure it directly. Let us instead minimize directly measured parameter estimation output error $\mathbf e_{\theta}$. 

To this end we introduce a cost function $J$:

\[
J = \frac{1}{2} \mathbf e_{\theta}^\top \mathbf e_{\theta}
\]

Expanding the definition, we get:

\[
J = \frac{1}{2} (\mathbf y - \mathbf M \Tilde{\theta})^\top (\mathbf y - \mathbf M \Tilde{\theta}) = 
\frac{1}{2} (\mathbf y^\top \mathbf y - 
2 \Tilde{\theta}^\top \mathbf M^\top \mathbf y + 
\Tilde{\theta}^\top \mathbf M^\top \mathbf M \Tilde{\theta})
\]

Derivative of $J$ with respect to parameters estimate is:
\[
\frac{\partial J}{\partial \Tilde{\theta}} = 
-\mathbf M^\top \mathbf y + 
\mathbf M^\top \mathbf M \Tilde{\theta}
\]

\end{flushleft}
\end{frame}


\begin{frame}{Least Squares in parameter estimation}
\framesubtitle{Solution}
\begin{flushleft}

We know that when the optimal estimation is found, the derivative $\frac{\partial J}{\partial \Tilde{\theta}}$ of the cost function will be equal to zero. Therefore:
\[
\frac{\partial J}{\partial \Tilde{\theta}} = 
-\mathbf M^\top \mathbf y + 
\mathbf M^\top \mathbf M \Tilde{\theta} = 0
\]
\[
 \Tilde{\theta} = (\mathbf M^\top \mathbf M)^{-1} \mathbf M^\top \mathbf y
\]

This presents the \emph{least squares solution} for the estimation problem.

\end{flushleft}
\end{frame}


\begin{frame}{Least Squares in parameter estimation}
\framesubtitle{Multiple measurements}
\begin{flushleft}

If the parameters $\theta$ are constant, we can use \emph{multiple measurements} to find them. Let us denote the value of $\mathbf M$ matrix for $i$-th measurement as $\mathbf M_i$ and the corresponding value of output vector $\mathbf y$ as $\mathbf y_i$. 

Then we can introduce compound output and estimation matrices for $n$ measurements:
\[
 \bar{\mathbf M} = \begin{bmatrix} \mathbf M_1 \\ ... \\ \mathbf M_n \end{bmatrix}
\]
\[
 \bar{\mathbf y} = \begin{bmatrix} \mathbf y_1 \\ ... \\ \mathbf y_n \end{bmatrix}
\]

Then we can use least squares to determine optimal parameter estimate same as before:
\[
 \Tilde{\theta} = (\bar{\mathbf M}^\top \bar{\mathbf M})^{-1} \bar{\mathbf M}^\top \bar{\mathbf y}
\]

\end{flushleft}
\end{frame}



\begin{frame}
\centerline{Lecture slides are available via Moodle.}
\bigskip
\centerline{You can help improve these slides at:}
\centerline{\url{https://github.com/SergeiSa/Linear-Control-Slides-Spring-2020}}
\bigskip
\centerline{Check Moodle for additional links, videos, textbook suggestions.}
\end{frame}

\end{document}
